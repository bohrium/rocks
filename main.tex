\documentclass[justified]{tufte-book}
\usepackage{amsmath, amssymb, amsthm, bm, moresize}
\usepackage{euler, ifthen, changepage}

\usepackage{graphicx, epsdice, xcolor, listings, float, wrapfig, caption}

\definecolor{mblu}{rgb}{0.05, 0.40, 0.70}

\geometry{
    left          =1.0in, % left margin
    textwidth     =5.0in, % main text block
    marginparsep  =0.2in, % gutter between main text block and margin notes
    marginparwidth=1.8in, % width of margin notes
}

\newcommand{\centeredgeometry}{
    \newgeometry{
        left          =1.0in, % left margin
        textwidth     =6.5in, % main text block
        marginparsep  =0.0in, % gutter between main text block and margin notes
        marginparwidth=0.0in, % width of margin notes
    }
}

\newcommand{\mtit}[3]{%
    \begin{adjustwidth}{0.0in}{-1.5in}
        \begin{center}
            \bf\sf
            {    \color{mblu}\vspace{0.0cm} \par \noindent \HUGE #1} 
            {\it \color{mblu}\vspace{0.3cm} \par \noindent \Huge #2}
            {\it             \vspace{0.3cm} \par \noindent \huge #3}
                             \vspace{0.6cm}
        \end{center}
    \end{adjustwidth}
}
\newcommand{\mchp}[1]{\let\clearpage\relax\chapter{\color{mblu}\textsf{#1}}}
\newcommand{\msec}[1]{\section{\color{mblu}\textsf{#1}}}

\usepackage{etoolbox}
\patchcmd{\thebibliography}{\section*{\refname}}{}{}{}

\newcommand{\EE}{\mathbb{E}}
\newcommand{\PP}{\mathbb{P}}
\newcommand{\RR}{\mathbb{R}}
%
\newcommand{\Dd}{\mathcal{D}}
\newcommand{\Ee}{\mathcal{E}}
\newcommand{\Ff}{\mathcal{F}}
\newcommand{\Gg}{\mathcal{G}}
\newcommand{\Hh}{\mathcal{H}}
\newcommand{\Ii}{\mathcal{I}}
\newcommand{\Kk}{\mathcal{K}}
\newcommand{\Ll}{\mathcal{L}}
\newcommand{\Mm}{\mathcal{M}}
\newcommand{\Nn}{\mathcal{N}}
\newcommand{\Pp}{\mathcal{P}}
\newcommand{\Ss}{\mathcal{S}}
\newcommand{\Tt}{\mathcal{T}}
\newcommand{\Uu}{\mathcal{U}}
\newcommand{\Vv}{\mathcal{V}}
\newcommand{\Xx}{\mathcal{X}}
\newcommand{\Yy}{\mathcal{Y}}

\newtheorem*{qst}{Question}
\newtheorem*{thm}{Theorem}
\newtheorem*{lem}{Lemma}
\newtheorem*{prop}{Proposition}
\newtheorem*{clm}{Claim}
\theoremstyle{definition}
\newtheorem*{dfn}{Definition}

\begin{document}

    \mtit{Rocks from the Ground Up}
         {A Firm Foundation in Geology}
         {Sam Tenka \hspace{1.0 cm} 2020}

    Rocks are hard; their study, harder.  These notes aim to survey rock theory
    not with the onomatapoeic impenetrability, weight, and dryness of museum
    glass, 3rd edition textbooks, and tasting sessions, but in basic terms that
    illustrate the unity of science.  The ideal audience is Emmy Noether and
    Henri Poincar\'e.  A brilliant high schooler might also enjoy these notes.

    \mchp{Scylla and Charybdis}

        Science is marked by its emphasis on observation --- above reason,
        aesthetics, intuition, and revelation from authority --- as a means to
        know the world.
        %
        Nevertheless, intuition plays a fundamental, unavoidable role in
        shaping our Bayesian priors.
        %
        Kant, for instance, proposed that our ability to sense gives rise to
        synthetic propositions known prior to experience.  To Kant, one's
        internal understanding of one's vision suggests before one sees any
        particular thing that space is three-dimensional.  One thus expects to
        observe nature as a collection of objects of definite color, smell, and
        texture, located in a three-dimensional space and evolving though
        one-dimensional time.
        %
        Shockingly, experiments in the 20th century revealed such coordinates
        of description to be far from independent, forcing a retreat from
        Kant's scene-setting intuitions.

        Specifically, \emph{relativity} denies us an absolute separation
        between space and time, for two may mix as easily as axes of space may
        rotate into each other.  A 30 mph cow chasing a 20 mph calf
        perceives its kid as approaching at \emph{more} than 10 mph.  

        Meanwhile, \emph{quantum mechanics} denies us a parameterization of
        objects by their observable properties.  Though via three separate
        experiments we may observe an apple's color, smell, or texture, to
        insist that the experiments could in principle be performed
        simultaneously --- and thus that the apple in principle has a definite
        color, smell, and texture --- leads to logical contradictions. 

        Relativity and quantum mechanics together suggest the (experimentally
        confirmed) possibility of particles of spin $1/2$ that necessarily
        obey \emph{Pauli exclusion}.  Such particles are the leading candidate
        for the building blocks of rocks.
        
        \msec{Classifying particles}
            An object's physical state inhabits the projective space $\Pp V$
            associated to some Hilbert space $V$.  The relativity group
            $SO(1;3)$ acts on Minkowski space $M$ and thus on $\Pp V$.  We
            assume this action is smooth and linear, i.e. described by a Lie
            algebra action $\rho:so(1;3)\to End(V)$.  Whenever $\exp(g)=id_M$
            we have $\exp(\rho(g)) = c_g \cdot id_V$ for some complex phase
            $c_g$.  Strikingly, the value of $c_g$ has bearing on experiments,
            and it is not always $1$.

        \msec{Relativistic quantum fields}
        \msec{Spin and statistics}
        \msec{The solidity of rocks}

    \mchp{Emma}
    \mchp{Frankenstein}
    \mchp{Ecclesiastes}
    \mchp{Zembla}


    %We limit prerequisites to \emph{calculus}, but these notes will reward
    %prior exposure to Ein\emph{stein}'s special relativity and its attendant
    %\emph{geo}metry.


\end{document}


